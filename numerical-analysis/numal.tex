\documentclass[12pt]{article}
\usepackage[margin=1in, headheight=20pt]{geometry}
\usepackage{xcolor}
\usepackage{tikz}
\usepackage{amsthm, amsmath, amssymb}
\usepackage{mathtools}
\usepackage[italicdiff]{physics}
\usepackage{enumitem}
\usepackage{lmodern}
\usepackage{fancyhdr}
\usepackage{pgfornament}

\definecolor{pagecolor}{HTML}{DCE2F0}
\definecolor{textcolor}{HTML}{373D4A}

\pagecolor{pagecolor}
\color{textcolor}

\pagestyle{fancy}
\fancyhf{}
\fancyhead[L]{Numerical Optimization}
\fancyhead[R]{\nouppercase{\leftmark}}
\fancyfoot[C]{}

\renewcommand{\headrule}{
  \vspace{-5pt}
  \hbox to \headwidth{
    \leaders\hrule height 0.5pt\hfill
    \hspace{5pt}
    \raisebox{0.20pt}{\pgfornament[width=1cm]{11}}
    \hspace{5pt}
    \leaders\hrule height 0.5pt\hfill
  }
}

\renewcommand{\footrule}{
  \vspace{-12pt}
  \hbox to \headwidth{
    \leaders\hrule height 3.5pt depth -3pt \hfill 
    \hspace{5pt} 
    \thepage 
    \hspace{5pt}
    \leaders\hrule height 3.5pt depth -3pt \hfill
  }
}

\fancypagestyle{plain}{
  \fancyhf{}
  \renewcommand{\headrulewidth}{0pt}
  \renewcommand{\headrule}{} 
  \fancyfoot[C]{}
  \renewcommand{\footrule}{
    \vspace{-12pt}
    \hbox to \headwidth{
      \rule[0.65ex]{0.47\headwidth}{0.5pt}%
      \hfill
      \thepage
      \hfill
      \rule[0.65ex]{0.47\headwidth}{0.5pt}%
    }
  }
}

\newcommand{\bb}[1]{\mathbb{#1}}
\newcommand{\cl}[1]{\mathcal{#1}}

\newcommand{\p}[1]{\left ( #1 \right )}
\newcommand{\bk}[1]{\left [ #1 \right ]}
\newcommand{\br}[1]{\left \{ #1 \}}
\newcommand{\ab}[1]{\langle #1 \rangle}

\newcommand{\f}[2]{\frac{#1}{#2}}
\newcommand{\nset}{\varnothing}
\newcommand{\oo}{\infty}
\DeclareMathOperator{\fl}{fl}
\newcommand{\NaN}{\texttt{NaN}}

\newcommand{\be}{\beta}
\newcommand{\gm}{\gamma}
\newcommand{\de}{\delta}
\newcommand{\De}{\Delta}
\newcommand{\ep}{\varepsilon}
\newcommand{\la}{\lambda}
\newcommand{\si}{\sigma}
\newcommand{\om}{\omega}
\newcommand{\Om}{\Omega}

\newcommand{\imp}{\Rightarrow}
\newcommand{\pmi}{\Leftarrow}
\renewcommand{\iff}{\Leftrightarrow}
\newcommand{\ffi}{\Rightarrow\!\Leftarrow}

\setlist[enumerate]{label=(\alph*)}

\newtheoremstyle{boldnote}
  {}
  {}
  {\itshape}
  {}
  {\bfseries}
  {.}
  { }
  {\thmname{#1}\thmnumber{ #2}\thmnote{ (\bfseries #3)}}
\theoremstyle{boldnote}
\newtheorem{theorem}{Theorem}[section]
\newtheorem{lemma}[theorem]{Lemma}

\theoremstyle{definition}
\newtheorem{definition}[theorem]{Definition}
\newtheorem{example}{Example}

\title{
    \textbf{Numerical Analysis} \\
}
\author{
    Dhyan Laad \\
    \texttt{2024ADPS0875G}
}
\date{}

\begin{document}
\maketitle

\section{Introduction to Computation}
\subsection{Floating Point Forms}
A real number may potentially have an infinite decimal expansion, but computers are limited by hardware, and as such store numbers with a terminating approximation.

\begin{definition}
    Given a real number $x$ with digits $d_1, d_2, \dots$, the \emph{$n$-digit, base $\beta$ floating point form}, or \emph{$n$-$\beta$ floating point form} is
    \[(-1)^s \times (0.d_1d_2\dots d_n)_\be \times \be^e\]
    where $s \in \{0, 1\}$ is the \emph{sign}, $e$ is the \emph{exponent}, and the $\be$-fraction
    \[(0.d_1d_2\dots d_n)_\be = \f{d_1}{\be^1} + \f{d_2}{\be^2} + \cdots + \f{d_n}{\be^n}\]
    is called the \emph{mantissa}. In the case that $d_1 \neq 0$, the representation is called the \emph{normalized floating point form}.
\end{definition}

For a fixed value of $\beta$ and $n$ as defined above, the notation $\fl (x)$ is used to denote the $n$-$\be$ floating point representation of $x$. Furthermore, for all computing systems, there are bounds on the values that the exponent $e$ can take. This leads to the concepts of underflow and overflow.

\begin{definition}
    Let a real number $x$ have a floating point form with exponent $e$. For a computing system with exponential range $(m , M)$ where $m$ and $M$ are integers,
    \begin{enumerate}
        \item if $e > M$, then the system is said to \emph{overflow}, and the result of the computation is denoted with a signed infinity: $\pm \oo$, and
        \item if $e < m$, then the system is said to \emph{underflow}, and the result of the computation is simply $0$.
    \end{enumerate}
\end{definition}

There are two ways to determine the mantissa of the floating point representation of a real number with more than $n$ digits: chopping and rounding. The chopped mantissa of the floating point representation of $x = 0.d_1d_2\dots d_nd_{n+1}\dots$ would simply be $(0.d_1d_2\dots d_n)$, while the rounded mantissa would be
\[\begin{cases}
    (0.d_1d_2\dots d_n) & d_{n+1} \in [0, \be/2), \\
    (0.d_1d_2\dots (d_n+1)) & d_{n+1} \in [\be/2, \be].
\end{cases}\]

\subsection{Errors}
The error of a floating point representation is a quantitification of how far removed it is from its true value.

\begin{definition}
    Let $x \in \bb R$. The \emph{absolute error} of its floating point representation is
    \[x - \fl(x).\]
\end{definition}

Note that since $\fl(x) \leq x$ for all $x \in \bb R$, the absolute error is always a positive quantity. Absolute error is the simplest quantitification but not the most useful, motivating a definition for relative error.

\begin{definition}
    The ratio of the absolute error to the true value of a real number $x$ is called its \emph{relative error}. It is customarily denoted with $\ep$:
    \[\ep = \f{x - \fl(x)}{x}.\]
\end{definition}

Another quantitification of how removed an approximation is from its true value is captured in the approximation's significant figures or significant digits.

\begin{definition}
    Let $x$ be a real number and $x^*$ be an approximation of it. Then if
    \[\abs{x - x^*} \leq \f 12 \be^{s-r+1}\]
    where $s$ is the largest integer such that $\be^s \leq \abs{x}$, then $x^*$ is said to approximate $x$ to $r$ \emph{significant figures} in $\be$.
\end{definition}

\begin{theorem}
    Let $\fl(x)$ be the $n$-$\be$ floating point representation for $x \in \bb R$, and set
    \[\ep = \f{x - \fl(x)}{x}.\]
    Then,
    \begin{enumerate}[font=\upshape]
        \item $\ep \leq \be^{-n+1}$ for chopped systems, and
        \item $\displaystyle \ep \leq \f 12 \be^{-n+1}$ for rounded systems.
    \end{enumerate}
\end{theorem}
\end{document}
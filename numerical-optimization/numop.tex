\documentclass[12pt]{article}
\usepackage[margin=1in, headheight=20pt]{geometry}
\usepackage{xcolor}
\usepackage{tikz}
\usepackage{pgfplots}
\usepackage{eso-pic}
\usepackage{amsthm, amsmath, amssymb}
\usepackage{mathtools}
\usepackage[italicdiff]{physics}
\usepackage{enumitem}
\usepackage{lmodern}
\usepackage{fancyhdr}
\usepackage{pgfornament}

\definecolor{pagecolor}{HTML}{DCE2F0}
\definecolor{textcolor}{HTML}{373D4A}

\pagecolor{pagecolor}
\color{textcolor}

\pagestyle{fancy}
\fancyhf{}
\fancyhead[L]{Numerical Optimization}
\fancyhead[R]{\nouppercase{\leftmark}}
\fancyfoot[C]{}

\renewcommand{\headrule}{
  \vspace{-5pt}
  \hbox to \headwidth{
    \leaders\hrule height 0.5pt\hfill
    \hspace{5pt}
    \raisebox{0.20pt}{\pgfornament[width=1cm]{11}}
    \hspace{5pt}
    \leaders\hrule height 0.5pt\hfill
  }
}

\renewcommand{\footrule}{
  \vspace{-12pt}
  \hbox to \headwidth{
    \leaders\hrule height 3.5pt depth -3pt \hfill 
    \hspace{5pt} 
    \thepage 
    \hspace{5pt}
    \leaders\hrule height 3.5pt depth -3pt \hfill
  }
}

\fancypagestyle{plain}{
  \fancyhf{}
  \renewcommand{\headrulewidth}{0pt}
  \renewcommand{\headrule}{} 
  \fancyfoot[C]{}
  \renewcommand{\footrule}{
    \vspace{-12pt}
    \hbox to \headwidth{
      \rule[0.65ex]{0.47\headwidth}{0.5pt}%
      \hfill
      \thepage
      \hfill
      \rule[0.65ex]{0.47\headwidth}{0.5pt}%
    }
  }
}

\newcommand{\bb}[1]{\mathbb{#1}}
\newcommand{\cl}[1]{\mathcal{#1}}

\newcommand{\p}[1]{\left ( #1 \right )}
\newcommand{\bk}[1]{\left [ #1 \right ]}
\newcommand{\br}[1]{\left \{ #1 \}}
\newcommand{\ab}[1]{\langle #1 \rangle}

\newcommand{\f}[2]{\frac{#1}{#2}}
\newcommand{\nset}{\varnothing}
\newcommand{\oo}{\infty}

\newcommand{\gm}{\gamma}
\newcommand{\de}{\delta}
\newcommand{\De}{\Delta}
\newcommand{\ep}{\varepsilon}
\newcommand{\la}{\lambda}
\newcommand{\si}{\sigma}
\newcommand{\om}{\omega}
\newcommand{\Om}{\Omega}

\newcommand{\imp}{\Rightarrow}
\newcommand{\pmi}{\Leftarrow}
\renewcommand{\iff}{\Leftrightarrow}
\newcommand{\ffi}{\Rightarrow\!\Leftarrow}

\setlist[enumerate]{label=(\alph*)}
\pgfplotsset{compat=newest}

\newtheorem{theorem}{Theorem}[section]
\newtheorem{lemma}[theorem]{Lemma}
\theoremstyle{definition}
\newtheorem{definition}[theorem]{Definition}
\newtheorem{example}{Example}

\title{
    \textbf{Numerical Optimization} \\
}
\author{
    Dhyan Laad \\
    \texttt{2024ADPS0875G}
}
\date{}

\begin{document}
\maketitle

\section{Review}
\subsection{Inner Product Spaces}
\begin{definition}
    Let $V$ be a real vector space. A function $\ab{\cdot, \cdot} : V \times V \to \bb R$ is called a \emph{real inner product} if it satisfies the following three properties for $x, y, z \in V$ and $c \in \bb R$:
    \begin{enumerate}
        \item $\ab{x, y} = \ab{y, x}$,
        \item $\ab{x + z, y} = \ab{x, y} + \ab{z, y}$,
        \item $\ab{cx, y} = c\ab{x, y}$, and
        \item $\ab{x, x} \geq 0$ and $\ab{x, x} = 0 \iff x = 0$.
    \end{enumerate}
\end{definition}

The inner product is the generalization of the dot product on Euclidean vector spaces: for $\vb x = (x_1, x_2, \dots , x_n)$ and $\vb y = (y_1, y_2, \dots , y_n)$ where $x_i, y_i \in \bb R$ for $i \in 1 : n$,
\[\vb x \cdot \vb y = \sum_{i=1}^n x_iy_i.\]
It may also be defined on $\bb C$-spaces, replacing (a) with conjugate symmetry: $\ab{x, y} = \overline{\ab{y, x}}$, and adding conjugate linearity in the second argument: $\ab{x, cy} = \bar{c}\ab{x, y}$.

\begin{definition}
    Let $V$ be a real vector space and $\ab{\cdot, \cdot}$ an inner product. Then, $(V, \ab{\cdot, \cdot})$ is called a \emph{real inner product space}.
\end{definition}

For brevity, we may simply state that $V$ is an inner product space, with the notation for the inner product being implicit.

\subsection{Normed Linear Spaces}
\begin{definition}
    Let $V$ be a real vector space. A function $\norm{\cdot} : V \to \bb R$ is called a \emph{norm} if it satisfies the following properties for $x, y \in V$ and $c \in \bb R$:
    \begin{enumerate}
        \item $\norm{cx} = \abs{c} \norm{x}$,
        \item $\norm{x} \geq 0$ and $\norm{x} = 0 \iff x = 0$, and
        \item $\norm{x + y} \leq \norm{x} + \norm{y}$.
    \end{enumerate}
\end{definition}

This last property is referred to as the \emph{triangle inequality}. The norm assigns a notion of length to vectors, and generalizes the standard formula for the length of a vector in a Euclidean vector space: for $\vb x = (x_1, x_2, \dots , x_n)$ where $x_i \in \bb R$ for $i \in 1 : n$.
\[\norm{\vb x} = \p{\sum_{i=1}^n x_i^2}^{1/2}.\]

\begin{definition}
    Let $V$ be a vector space and $\norm{\cdot}$ a norm. Then $(V, \norm{\cdot})$ is called a \emph{normed linear space}.
\end{definition}

Once again, we may conventionally ommit the norm from notation when defining a new normed linear space.

\begin{definition}
    Let $p \geq 1$. The \emph{$p$-norm} (or \emph{$\ell^p$-norm}) of a vector $\vb x = (x_1, x_2, \dots, x_n)$ where $x_i \in \bb R$ for $i \in 1 : n$ is
    \[\norm{\vb x}_p \coloneqq \p{\sum_{i=1}^n \abs{x_i}^p}^{1/p}.\]
\end{definition}

For $p=1$, we get the \emph{taxicab} or \emph{Manhattan} norm, for $p=2$, we get the standard Euclidean norm, and for $p \to \oo$, the $p$-norm approaches the \emph{infinity} or \emph{maximum} norm:
\[\norm{\vb x}_\oo \coloneqq \max_{i} \abs{x_i}.\]
For $p \in (0, 1)$, the triangle inequality does not hold, and the resulting functions are called \emph{quasinorms}.

\end{document}
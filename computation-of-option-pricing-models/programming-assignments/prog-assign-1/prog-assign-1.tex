\documentclass[12pt]{article}
\usepackage[margin=1in, headheight=20pt]{geometry}
\usepackage{graphicx}
\usepackage{amsthm, amsmath, amssymb}
\usepackage{lmodern}
\usepackage{parskip}
\usepackage{listings}
\usepackage{xcolor}

\definecolor{mygreen}{RGB}{28,172,0}
\definecolor{mylilas}{RGB}{170,55,241}

\definecolor{mygreen}{RGB}{28,172,0}
\definecolor{mylilas}{RGB}{170,55,241}
\definecolor{lightgray}{gray}{0.98}

\lstset{
    language=Matlab,
    backgroundcolor=\color{lightgray},
    frame=single,
    rulecolor=\color{black},
    basicstyle=\ttfamily\small,
    breaklines=true,
    keywordstyle=\color{blue},
    commentstyle=\color{mygreen},
    stringstyle=\color{mylilas},
    numbers=left,
    numberstyle={\tiny \color{black}},
    stepnumber=1,
    numbersep=9pt,
    showstringspaces=false
}

\title{
    \textbf{Computation of Option Pricing Models} \\
    \textbf{Programming Assignment 1}
}
\author{
  Dhyan Laad \\
  \texttt{2024ADPS0875G}
}
\date{}

\begin{document}
\maketitle

\begin{enumerate}
  \item This script calculates and plots the payoff for two call options. It sets two strike prices ($\$ 2$ and $\$ 4$) and evaluates the result across a range of asset values from 0 to 6. The calculation subtracts the value of a call option at the higher price from one at the lower price, creating a graph that starts at zero, rises linearly between the two thresholds, and then caps at a fixed maximum value.
  \begin{center}
    \includegraphics[width=0.375\textwidth]{fig1.pdf}
  \end{center}

  \item This script compares the growth of an investment using two different methods: monthly compound interest and continuous compound interest. It defines an initial value of 5 and an interest rate of 15\% over a 5-year period. The calculation generates discrete data points for monthly compounding and overlays them with a smooth curve representing continuous compounding to visualize the relationship between the two growth models.
  \begin{center}
    \includegraphics[width=0.375\textwidth]{fig2.pdf}
  \end{center}

  \item This script generates a 3D visualization of the normal distribution probability density function. It sets the mean to zero and evaluates the distribution across a grid where the standard deviation varies from 1 to 5. The calculation computes the density values for each combination, and the resulting waterfall plot illustrates how the bell curve flattens and spreads out as the standard deviation increases.
  \begin{center}
    \includegraphics[width=0.375\textwidth]{fig3.pdf}
  \end{center}

  What follows is the \textsc{Matlab} code for the exponential density function:
  \[f(x) =
  \begin{cases}
    e^{-\lambda x} & x > 0, \\
    0 & x \leq 0.
  \end{cases}
  \]

  \begin{lstlisting}
    clf

    dlambda = 0.25;
    dx = 0.1;
    [X, LAMBDA] = meshgrid(0:dx:6, 0.5:dlambda:3);

    f(x) = lambda * e^(-lambda * x)
    Z = LAMBDA .* exp(-LAMBDA .* X);

    waterfall(X, LAMBDA, Z)
    xlabel('x')
    ylabel('\lambda')
    zlabel('f(x)')
    title('Exponential density for various \lambda')
  \end{lstlisting}

  And the associated plot.

  \begin{center}
    \includegraphics[width=0.375\textwidth]{fig4.pdf}
  \end{center}

  \item This script illustrates the central limit theorem (CLT) using a Monte Carlo simulation. It performs $M=10,000$ experiments, where each experiment calculates the standardized sum of $n=500$ independent random variables drawn from a non-normal distribution. The code aggregates these standardized sums into vector $S$ and plots a normalized histogram to represent the empirical probability density. Finally, it superimposes the theoretical standard normal density function, $\mathcal{N}(0,1)$, as a red curve to visually demonstrate that the distribution of the sums converges to a normal distribution.
  
   \begin{center}
    \includegraphics[width=0.375\textwidth]{fig5.pdf}
  \end{center}
  

\end{enumerate}
\end{document}